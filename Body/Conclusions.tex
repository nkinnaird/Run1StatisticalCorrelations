%!TEX root = ../StatisticalCorrelations.tex

\graphicspath{{Body/Figures/}}

\clearpage
\section{Conclusions}


Correlation coefficients were determined between analyses and methods for the different \Rone \wa analyzers and datasets, using a Monte Carlo with real data input. These correlation coefficients and variations of them were used in different approaches to the \wa combination, and associated results will be detailed in a forthcoming note by D. Sweigart \cite{CombinationMeeting}\footnote{The favored approach is a staged averaging method, using no correlation coefficients in the combination. An error on the combination value is then determined using different combination methods using the provided correlation coefficients in a variety of ways.}. These correlations and the Monte Carlo provide a starting point for any combination effort of future analyses of Run~2 and beyond. 

While some determined correlations did exceed the conditions for the high-correlation regime, in general the correlations were very consistent across datasets and methods. When comparing the Monte Carlo used to generate the pseudo-data to real data, there are a number of facets where differences might arise from, as well as places where the Monte Carlo could be improved:
\begin{itemize}
	\item{There are systematic differences between analyses, both in how various systematics affect the best-fit \R values (eg. less effect in the R-Method), and how analyses correct and estimate the effects. The errors from the systematic effects are not included in the calculation of the correlations here.}
	\item{One of the main systematic differences in the analyses is the use of different pileup methods. Including the effect of pileup should be sub-leading since it is the corrected time spectrum that is of most importance, and different energy thresholds have a greater effect on the number of counts. It would be quite an undertaking to implement multiple pileup methods into the Monte Carlo, and probably not worth the effort.}
	\item{The Q-Method is a different analysis method completely, and it's differences won't be adequately modeled here. Since the correlations are so much less than the other datasets it's not so much of a concern, but it should be kept in mind. The larger errors on the Q-Method correlations in part reflect this.}
	\item{The \ROOT \texttt{TF2}s used to generate the data could use finer time and energy points (with a different implementation), so that the hit generation wouldn't introduce aliasing frequencies in some of the Q-Method fits.}
	\item{The \RE vs \RW comparison was based solely on clusters from a single \RW analysis, A. Fienberg's. The comparison could be done with the hits from other \RW analyzers, though the differences are expected to be small. The comparison was also made on non-pileup corrected counts.}
	\item{Energy bin functions for \RW were provided by a single \RW analyzer, M. Sorbara. Other analyzers most likely have small differences.}
	\item{The analyses to data randomize out the fast rotation by randomizing times at the cyclotron period, and this effect is not included in the Monte Carlo. Different analyses then use different numbers of random seeds, and some analyzers quote mean \R values from from fits to the seeds while others quote the closest \R value to the mean. Differences in results due to these choices would increase the amount of expected statistical deviation slightly. Generating data corresponding to a number of random seeds specific to every analyzer would make the computation time needed increase drastically, and probably wouldn't be feasible.}
	\item{The BU analysis effort randomized out the vertical waist effect. This increases the error on the best-fit \R values by \SIrange{10}{30}{ppb} depending on dataset, and would result in slightly different (perhaps negligibly) correlation coefficients.}
	\item{The \texttt{TRandomMixMax} randomization class may be insufficient as shown by the non-flat R-Method p value distribution. A better random number generator may be used, such as \texttt{Ranlux64}, however there is the downside that the computation time would increase significantly.}
\end{itemize}
The effects of each of these points would in general decrease the correlations between the analyses, as the differences between them increase. It is possible though not guaranteed that implementing the respective changes would push the correlation coefficients below the high-correlation regime cut-off. Some of the points may be straightforwardly improved at the cost of computation time, while others would take significant work for most likely relatively little gain. Whether it is worth improving the correlation coefficients or not for future runs will depend on the combination approaches adopted. For now the correlation coefficients presented here are the most accurate ones available for the experiment.


